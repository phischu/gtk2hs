% Gtk2Hs-DG.tex
\begin{hcarentry}[updated]{Gtk2Hs}
\label{gtk2hs}
\report{Daniel Wagner}%05/13
\participants{Axel Simon, Duncan Coutts, Andy Stewart, and many others}
\status{beta, actively developed}
\makeheader

Gtk2Hs is a set of Haskell bindings to many of the libraries included
in the Gtk+/Gnome platform. Gtk+ is an extensive and mature
multi-platform toolkit for creating graphical user interfaces.

GUIs written using Gtk2Hs use themes to resemble the native look on
Windows. Gtk is the toolkit used by Gnome, one of the two major GUI toolkits
on Linux. On Mac OS programs written using Gtk2Hs are run by Apple's
X11 server but may also be linked against a native Aqua implementation
of Gtk.

\Separate
Gtk2Hs features:
\begin{itemize}
\item Automatic memory management (unlike some other C/C++ GUI
libraries, Gtk+ provides proper support for garbage-collected languages)
\item Unicode support
\item High quality vector graphics using Cairo
\item Extensive reference documentation
\item An implementation of the ``Haskell School of Expression'' graphics
API
\item Bindings to many other libraries that build on Gtk: gio, GConf,
  GtkSourceView 2.0, glade, gstreamer, vte, webkit
\end{itemize}

\Separate

Since the last release, there have been many bugfixes, and Peter Davies and
Hamish Mackenzie have begun adding experimental Gtk3 support, enabled by the
``gtk3'' cabal flag.

\FurtherReading
\begin{compactitem}
\item News and downloads:
  \url{http://haskell.org/gtk2hs/}

\item Development version:
  \texttt{darcs get} \url{http://code.haskell.org/gtk2hs/}
\end{compactitem}
\end{hcarentry} 
